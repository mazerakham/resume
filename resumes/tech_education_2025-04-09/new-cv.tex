\documentclass[11pt]{article}

% Basic packages
\usepackage[utf8]{inputenc}
\usepackage[T1]{fontenc}
\usepackage{geometry}
\usepackage{hyperref}
\usepackage{enumitem}
\usepackage{fontawesome5}
\usepackage{graphicx}
\usepackage{color}
\usepackage{titlesec}

% Page geometry
\geometry{a4paper, margin=2cm}

% Colors
\definecolor{headingColor}{RGB}{0, 0, 128}

% Section formatting
\titleformat{\section}
  {\Large\bfseries\color{headingColor}}
  {}{0em}{}[\titlerule]
\titlespacing{\section}{0pt}{12pt}{6pt}

% Subsection formatting
\titleformat{\subsection}
  {\large\bfseries}
  {}{0em}{}
\titlespacing{\subsection}{0pt}{8pt}{4pt}

% Custom commands for CV entries
\newcommand{\cventry}[3]{
  \subsection[#1]{#1 \hfill #2}
  #3
}

% Header information
\newcommand{\cvheader}{
  \begin{center}
    {\Huge\bfseries Jake Mirra, Ph.D.}\\[0.5em]
    \faEnvelope\ \href{mailto:jmirra1515@gmail.com}{\texttt{jmirra1515@gmail.com}} $\cdot$
    \faPhone\ \texttt{(412) 944-7767} $\cdot$
    \faMapMarker\ \texttt{Austin, TX}\\
    \faHome\ \texttt{7708 San Felipe Blvd, Unit 12, Austin, TX 78729}
  \end{center}
}

\begin{document}

\cvheader

\section{Education}
\cventry{Ph.D. in Mathematics}{September 2012 -- June 2018}{
  \textbf{University of Pittsburgh, Main}\\
  Thesis title: \emph{Hölder Continuous Mappings into Sub-Riemannian Manifolds (2018).}\\
  Specialized in geometric analysis and sub-Riemannian geometry.\\
  \textbf{Teaching Experience:}
  \begin{itemize}[leftmargin=*]
    \item Primary instructor for Differential Equations and Linear Algebra courses for 4 semesters
    \item Teaching Assistant for 6 semesters, providing classroom instruction and assessment
    \item Developed comprehensive course materials and innovative teaching methodologies
    \item Received consistently positive student evaluations for clarity of instruction
  \end{itemize}
  \textbf{Research Publications:}
  \begin{itemize}[leftmargin=*]
    \item Published research in the field of geometric analysis in peer-reviewed journals
    \item Presented research findings at departmental seminars and conferences
    \item Collaborated with faculty on research projects in sub-Riemannian geometry
  \end{itemize}
}

\cventry{B.S. in Mathematics}{September 2011 -- September 2012}{
  \textbf{University of Pittsburgh, Main}\\
  Minor in Computer Science, completed accelerated program with honors.
}

\section{Employment History}
\cventry{Curriculum Developer}{2018}{
  \textbf{Dulwich College International via Redhat}
  \begin{itemize}[leftmargin=*]
    \item Developed comprehensive late middle school honors mathematics curriculum
    \item Applied pedagogical expertise to create engaging and challenging course materials
    \item Designed curriculum to foster critical thinking and mathematical reasoning skills
    \item Incorporated modern educational approaches to enhance student learning outcomes
  \end{itemize}
}

\cventry{Lecturer and Teaching Assistant}{September 2012 -- June 2018}{
  \textbf{University of Pittsburgh, Department of Mathematics}
  \begin{itemize}[leftmargin=*]
    \item Taught Differential Equations and Linear Algebra lectures as primary instructor for 4 semesters
    \item Served as Teaching Assistant for 6 semesters, teaching students and grading assignments
    \item Developed course materials and assessment strategies to enhance student learning
    \item Provided individualized support to students during office hours
    \item Collaborated with faculty on course development and improvement
  \end{itemize}
}

\cventry{Graduate Student}{September 2012 -- June 2018}{
  \textbf{University of Pittsburgh, Department of Mathematics}
  \begin{itemize}[leftmargin=*]
    \item Conducted research in geometric analysis and sub-Riemannian geometry
    \item Accumulated over 3,000 hours of tutoring experience across diverse mathematical subjects
    \item Innovated in Differential Equations and Linear Algebra lectures, publishing YouTube content and Mathematica-based labs
    \item Enabled students to progress further through theory and applications than in traditionally-taught undergraduate classes
    \item Received outstanding student testimonials for innovative teaching methods
  \end{itemize}
}

\cventry{Senior Software Engineer}{March 2022 -- September 2024}{
  \textbf{Ender, Austin}
  \begin{itemize}[leftmargin=*]
    \item Mentored and grew the development team, fostering a collaborative learning environment
    \item Applied analytical and problem-solving skills from mathematical background to complex technical challenges
    \item Developed leadership experience through cross-functional collaboration and team management
  \end{itemize}
}

\cventry{Software Engineer}{July 2019 -- March 2022}{
  \textbf{Ender, Austin}
  \begin{itemize}[leftmargin=*]
    \item Founding engineer at a property management software company
    \item Applied mathematical and analytical thinking to software architecture and development
  \end{itemize}
}

\section{Skills}
\begin{itemize}[leftmargin=*]
  \item \textbf{Programming Languages:} Python, JavaScript, TypeScript, SQL, C++, Java
  \item \textbf{Frameworks \& Libraries:} React, Node.js, Django, Flask, TensorFlow, PyTorch
  \item \textbf{Tools \& Technologies:} Git, Docker, AWS, GCP, CI/CD, Kubernetes
  \item \textbf{Mathematics:} Geometric Analysis, Sub-Riemannian Geometry, Differential Equations, Linear Algebra
  \item \textbf{Teaching:} Curriculum Development, Instructional Design, Assessment Creation, Student Mentoring
\end{itemize}

\section{Publications}
\begin{itemize}[leftmargin=*]
  \item Mirra, J. (2018). \emph{Hölder Continuous Mappings into Sub-Riemannian Manifolds}. Ph.D. Thesis, University of Pittsburgh.
  \item Hajłasz, P. \& Mirra, J. (2013). "The Lusin Theorem and Horizontal Graphs in the Heisenberg Group." \emph{Analysis and Geometry in Metric Spaces}, 1, 295-301.
\end{itemize}

\section{Interests}
\begin{itemize}[leftmargin=*]
  \item Classical piano, chess, tutoring
\end{itemize}

\section{Testimonials}
\begin{itemize}[leftmargin=*]
  \item "I liked being able to use Mathematica for simple integrals and other computations that seemed too hard for paper and pencil. It seemed as though I was able to complete problems of a higher complexity that I wouldn't have even attempted without the program. (Differential Equations student)"
  \item "I enjoy the more modern and practical approach of utilizing software for the computations. It allowed me to clearly see the linear algebra concepts and not be distracted by tedious computation. (Linear Algebra student)"
\end{itemize}

\end{document}
